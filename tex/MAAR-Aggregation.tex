\vspace{-2pt}
\subsection{Fair Aggregation across Many Apps}
\label{subsec:aggregation}

MAAR DSE needs to aggregate the normalized application efficiencies to a single fitness value to judge platforms quantitatively. This section proposes multiple aggregation options with mathematics analyses.

Illustrating the aggregation challenge, \figref{fig:eff} shows relative efficiencies of application on an example platform (blue) normalized by the lower bound SW platform (orange) in \figref{fig:effSW} and by the upper bound OOP (grey) in \figref{fig:effOOP}. In \figref{fig:effSW}, the relative efficiency of SW is always equal to 1, because it is normalized by itself. The OOP has higher $rEFF_{SW}$ than the MAAR platform for each application because OOP is the optimal platform design for each application. \figref{fig:effOOP} shows relative efficiency compared with OOP $rEFF_{OOP}$ for SW, MAAR platform and OOP. Similarly, OOP always has a $rEFF_{OOP}$ of 1 because it is normalized by itself. MAAR platform has $rEFF_{OOP}$ between SW and OOP, because MAAR accelerates some kernels to achieve better performance than SW, and is not the application own optimal platform, so it performs worse than the OOP.



\subsubsection{Average Aggregation}

To judge the fitness of a platform for all applications, the intuitive method is to aggregate the overall efficiency increase or loss across all applications.
In \figref{fig:effSW}, a MAAR platform with a more blue area (or a less grey area) is better, because it has a more overall increase (or a less loss). 
To achieve the best MAAR platform $p_{X}$, DSE should maximize the blue area ($Area_{b}$) or minimize the grey area ($Area_{g}$). 
Assuming the length of total number of applications in y-axis is equal to 1, 
$Area_{b}$ is equal to $\sum_{i=0}^{\#a} ( rEFF_{SW}(a_{i}, p_{X}) -  rEFF_{SW}(a_{i}, p_{SW}) ) / \#a $, and $Area_{g}$ is $\sum_{i=0}^{\#a} ( rEFF_{SW}(a_{i}, p_{OOP}) -  rEFF_{SW}(a_{i}, p_{X}) ) / \#a $.
After simplification, both maximizing $Area_b$ and minimizing $Area_g$ become equivalent to maximizing $\sum_{i=0}^{\#a} rEFF_{SW}(a_{i}, p_{X}) / \#a$, which is the average efficiency improvement of platform $p_{X}$.  

Both methods are equivalent because of the SW and OOP platforms are fixed for a set of applications, so the summed SW and OOP efficiency areas are constant. 
Therefore, maximizing the efficiency difference between platform $p_{X}$ and SW, or minimizing the difference between $p_{X}$ and OOP leads to the same result, which is to maximize the average efficiency improvement of $p_{X}$. 
The average of $rEEF_{SW}$ ($A \mhyphen SW$) across all applications becomes the first MAAR DSE fitness function, which is represented in the first line of Eq.~\eqref{eq:avg}.

\vspace{-8pt}
\begin{equation}
\begin{split}
	A\mhyphen SW (p_{X}) &= \sum_{i=0}^{\#a} rEFF_{SW}(a_{i}, p_{X}) / \#a \\
	A\mhyphen OOP (p_{X}) &= \sum_{i=0}^{\#a} rEFF_{OOP}(a_{i}, p_{X}) / \#a
\label{eq:avg}
\end{split}
\end{equation}

Similarly in \figref{fig:effOOP}, maximizing the difference in $rEFF_{OOP}$ between MAAR platform $p_{X}$ and SW, and minimizing the difference between $p_{X}$ and OOP are both equivalent to maximize the average efficiency achievement $rEFF_{OOP}$, as displayed in the second line of Eq.~\eqref{eq:avg}. $A \mhyphen OOP$ is the second proposed fitness function.   



\subsubsection{Log Area Aggregation}

In previous average aggregation, high relative efficiency applications have more effect on fitness. The value difference between high and low relative efficiency across applications is still a little significant, especially in $rEFF_{SW}$. To shrink the difference among applications and pay more attention to low-efficiency applications, a new aggregation with logarithm is proposed.
Instead of maximizing/minimizing the relative efficiency difference between MAAR platform and SW/OOP, the new method ($Alog$) first takes the logarithm of the efficiency to shrink the difference in relative efficiency among applications, then maximizes/minimizes the resulting logarithmic area.

\begingroup\makeatletter\def\f@size{7}\check@mathfonts
\vspace{-8pt}
\begin{equation}
\begin{split}
	Alog\mhyphen SW(p_{X})
	&= \sum_{i=0}^{\#a} ( \log rEFF_{SW}(a_{i}, p_{X}) - \log rEFF_{SW}(a_{i}, p_{SW}) ) / \#a \\
	&= \sum_{i=0}^{\#a} ( \log rEFF_{SW}(a_{i}, p_{X}) ) / \#a \\
	%&= \sum_{i=0}^{\#a} ( \log EFF(a_{i}, p_{X}) - \log EFF(a_{i}, p_{SW}) - \log 1 ) / \#a \\
	&= \sum_{i=0}^{\#a} ( \log EFF(a_{i}, p_{X}) ) / \#a - \sum_{i=0}^{\#a} ( \log EFF(a_i, p_{SW}) ) / \#a
\label{eq:areaSW}
\end{split}
\end{equation}
\endgroup

Eq.~\eqref{eq:areaSW} shows the the log area of the efficiency improvement $Alog\mhyphen SW$. First, note that $\log rEFF_{SW}(a_i, p_{SW}) = 0$ for all $a_i$. Next, since the efficiency of an application in software $EFF(a_i, p_{SW})$ is constant for all applications, it does not affect which platform $Alog\mhyphen SW$ is maximized for. The same process can be followed for the log area of achievement $Alog\mhyphen OOP$. Therefore, the normalization factor becomes unimportant in log area, so $Alog\mhyphen SW$ and $Alog\mhyphen OOP$ become equivalent to the log area of efficiency as shown in Eq.~\eqref{eq:area}. The fact that the normalization factor ``disapears'' in $Alog$ suggests that taking the logarithm of the efficiency of an application removes its dependence on the application's potential for efficiency without the need for a specific normalization. 


\begingroup\makeatletter\def\f@size{9}\check@mathfonts
\vspace{-8pt}
\begin{equation}
\begin{split}
	&Alog\mhyphen SW(p_{X}) \equiv Alog\mhyphen OOP(p_{X}) \\
	&\equiv Alog(p_{X}) = \sum_{i=0}^{\#a} ( \log EFF(a_{i}, p_{X}) ) / \#a	
\end{split}
\label{eq:area}
\end{equation}
\endgroup

\subsubsection{Median Aggregation}

Instead of focusing on the overall relative efficiency of all applications, maximizing the median efficiency across applications is another fitness function choice. 
Because median-maximizing behavior is more resilient to outliers, it is a good candidate to balance acceleration potential across many applications, and the median-maximizing guided DSE may provide a more balanced platform with a good efficiency distribution of applications. 
Eq.~\eqref{eq:med} shows two median maximizing fitness functions of two relative efficiencies $rEFF_{SW}$ and $rEFF_{OOP}$. 

\vspace{-4pt}
\begin{equation}
\begin{split}
	M\mhyphen SW (p_{X}) &= med(rEFF_{SW}(a_{i}, p_{X})),   i = 0,1,...,\#a \\
	M\mhyphen OOP (p_{X}) &= med(rEFF_{OOP}(a_{i}, p_{X})), i = 0,1,...,\#a
\label{eq:med}
\end{split}
\end{equation}